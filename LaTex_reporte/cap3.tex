\section{Conclusiones}
\begin{itemize}
    \item La Resolución JM-47-2022 establece un marco regulatorio robusto en Guatemala, que mejora la gestión del riesgo de crédito bancario mediante la adopción de prácticas más rigurosas y estructuradas para la valoración y clasificación de créditos, promoviendo así la estabilidad y la sostenibilidad financiera en el sector.
    \item El modelo matemático introducido en esta investigación ofrece un enfoque dual para la regulación de precios en el sector bancario de Guatemala. Funciona, en primer lugar, como una herramienta de supervisión de precios por una autoridad reguladora, destinada a prevenir que los bancos establezcan tarifas excesivamente altas. Simultáneamente, permite a los bancos ajustar sus propios precios de manera que puedan asegurar su rentabilidad, formar reservas adecuadas y evitar saturar a los clientes, lo cual podría derivar en incumplimientos de pago. Así, este modelo facilita tanto el establecimiento de precios justos como la gestión de riesgos.
    \item El teorema de existencia de la función de personalización de precios representa un avance significativo en el estudio de los precios de créditos en Guatemala. Este planteamiento garantiza que es posible asignar un precio específico a un crédito al emplear el modelo propuesto en esta investigación. De este modo, se introduce una nueva metodología que modifica el enfoque tradicional hacia la fijación de precios en el sector crediticio, ofreciendo una perspectiva más adaptada y precisa para cada situación particular.
\end{itemize}