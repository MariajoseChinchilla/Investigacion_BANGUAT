\section{Anexos}

\subsection{Anexo I}
\begin{table}[H]
\centering
%\caption*{\textbf{Probabilidad de Incumplimiento para diferentes subsegmentos}}
\begin{subtable}[t]{0.32\textwidth}
\centering
\begin{tabular}{@{}lc@{}}
\toprule
\textbf{Categoría} & \textbf{Probabilidad (\%)} \\
\midrule
A & 2.6 \\
B & 8.0 \\
C & 17.7 \\
D & 29.8 \\
E & 100.0 \\
\bottomrule
\end{tabular}
\caption{\textbf{Subsegmento de comercio}}
\end{subtable}
\hfill
\begin{subtable}[t]{0.32\textwidth}
\centering
\begin{tabular}{@{}lc@{}}
\toprule
\textbf{Categoría} & \textbf{Probabilidad (\%)} \\
\midrule
A & 1.8 \\
B & 5.3 \\
C & 7.8 \\
D & 21.2 \\
E & 100.0 \\
\bottomrule
\end{tabular}
\caption{\textbf{Subsegmento de industrias manufactureras}}
\end{subtable}
\hfill
\begin{subtable}[t]{0.32\textwidth}
\centering
\begin{tabular}{@{}lc@{}}
\toprule
\textbf{Categoría} & \textbf{Probabilidad (\%)} \\
\midrule
A & 4.3 \\
B & 11.0 \\
C & 16.0 \\
D & 39.4 \\
E & 100.0 \\
\bottomrule
\end{tabular}
\caption{\textbf{Subsegmento de actividades inmobiliarias y construcción}}
\end{subtable}

% Continuación de las tablas en la siguiente fila
\begin{subtable}[t]{0.32\textwidth}
\centering
\begin{tabular}{@{}lc@{}}
\toprule
\textbf{Categoría} & \textbf{Probabilidad (\%)} \\
\midrule
A & 1.4 \\
B & 4.2 \\
C & 6.0 \\
D & 20.0 \\
E & 100.0 \\
\bottomrule
\end{tabular}
\caption{\textbf{Subsegmento de suministro de electricidad, gas y agua}}
\end{subtable}
\hfill
\begin{subtable}[t]{0.32\textwidth}
\centering
\begin{tabular}{@{}lc@{}}
\toprule
\textbf{Categoría} & \textbf{Probabilidad (\%)} \\
\midrule
A & 2.6 \\
B & 7.0 \\
C & 10.2 \\
D & 25.5 \\
E & 100.0 \\
\bottomrule
\end{tabular}
\caption{\textbf{Subsegmento de establecimientos financieros}}
\end{subtable}
\hfill
\begin{subtable}[t]{0.32\textwidth}
\centering
\begin{tabular}{@{}lc@{}}
\toprule
\textbf{Categoría} & \textbf{Probabilidad (\%)} \\
\midrule
A & 4.7 \\
B & 9.6 \\
C & 14.6 \\
D & 35.7 \\
E & 100.0 \\
\bottomrule
\end{tabular}
\caption{\textbf{Subsegmento de agricultura, ganadería, silvicultura y pesca}}
\end{subtable}

% ... Continuación de las tablas anteriores

\begin{subtable}[t]{0.32\textwidth}
\centering
\begin{tabular}{@{}lc@{}}
\toprule
\textbf{Categoría} & \textbf{Probabilidad (\%)} \\
\midrule
A & 4.0 \\
B & 11.0 \\
C & 17.6 \\
D & 47.6 \\
E & 100.0 \\
\bottomrule
\end{tabular}
\caption{\textbf{Subsegmento de servicios y otros}}
\end{subtable}
\hfill
\begin{subtable}[t]{0.32\textwidth}
\centering
\begin{tabular}{@{}lc@{}}
\toprule
\textbf{Categoría} & \textbf{Probabilidad (\%)} \\
\midrule
A & 5.7 \\
B & 11.0 \\
C & 18.7 \\
D & 54.0 \\
E & 100.0 \\
\bottomrule
\end{tabular}
\caption{\textbf{Subsegmento de comercio}}
\end{subtable}
\hfill
\begin{subtable}[t]{0.32\textwidth}
\centering
\begin{tabular}{@{}lc@{}}
\toprule
\textbf{Categoría} & \textbf{Probabilidad (\%)} \\
\midrule
A & 6.8 \\
B & 11.5 \\
C & 16.7 \\
D & 40.2 \\
E & 100.0 \\
\bottomrule
\end{tabular}
\caption{\textbf{Subsegmento de servicios y otros}}
\end{subtable}
\end{table}\label{categorias}

\begin{table}[H]
\centering
% Continuación de las tablas en la siguiente fila
\begin{subtable}[t]{0.32\textwidth}
\centering
\begin{tabular}{@{}lc@{}}
\toprule
\textbf{Categoría} & \textbf{Probabilidad (\%)} \\
\midrule
A & 4.3 \\
B & 6.7 \\
C & 8.5 \\
D & 17.3 \\
E & 100.0 \\
\bottomrule
\end{tabular}
\caption{\textbf{Subsegmento de hipotecarios para vivienda}}
\end{subtable}
\hfill
\begin{subtable}[t]{0.32\textwidth}
\centering
\begin{tabular}{@{}lc@{}}
\toprule
\textbf{Categoría} & \textbf{Probabilidad (\%)} \\
\midrule
A & 7.0 \\
B & 9.8 \\
C & 24.6 \\
D & 61.0 \\
E & 100.0 \\
\bottomrule
\end{tabular}
\caption{\textbf{Subsegmento de cédulas hipotecarias}}
\end{subtable}
\hfill
\begin{subtable}[t]{0.32\textwidth}
\centering
\begin{tabular}{@{}lc@{}}
\toprule
\textbf{Categoría} & \textbf{Probabilidad (\%)} \\
\midrule
A & 5.0 \\
B & 16.5 \\
C & 31.0 \\
D & 68.0 \\
E & 100.0 \\
\bottomrule
\end{tabular}
\caption{\textbf{Subsegmento de tarjeta de crédito}}
\end{subtable}

% Continuación de las tablas en la siguiente fila
\begin{subtable}[t]{0.32\textwidth}
\centering
\begin{tabular}{@{}lc@{}}
\toprule
\textbf{Categoría} & \textbf{Probabilidad (\%)} \\
\midrule
A & 4.0 \\
B & 8.2 \\
C & 12.4 \\
D & 41.4 \\
E & 100.0 \\
\bottomrule
\end{tabular}
\caption{\textbf{Subsegmento de vehículos}}
\end{subtable}
\hfill
\begin{subtable}[t]{0.32\textwidth}
\centering
\begin{tabular}{@{}lc@{}}
\toprule
\textbf{Categoría} & \textbf{Probabilidad (\%)} \\
\midrule
A & 3.6 \\
B & 8.6 \\
C & 15.6 \\
D & 32.5 \\
E & 100.0 \\
\bottomrule
\end{tabular}
\caption{\textbf{Subsegmento fiduciario}}
\end{subtable}
\hfill
\end{table}

\subsection{Anexo II}
\begin{itemize}\label{codigo sarima}
  \item Importar las librerías necesarias.
\begin{lstlisting}
import pandas as pd
import matplotlib.pyplot as plt
import seaborn as sns
from pmdarima.arima import auto_arima
from statsmodels.tsa.statespace.sarimax import SARIMAX
\end{lstlisting}

  \item Cargar los datos de las reservas y el ritmo inflacionario.
\begin{lstlisting}
reservas = pd.read_excel("db/datos_reservas.xlsx")
\end{lstlisting}

  \item Crear un DataFrame con un índice de tiempo mensual desde enero de 2018 hasta diciembre de 2023.
\begin{lstlisting}
fecha_inicio = '2018-01-01'
fecha_fin = '2023-12-01'
fechas = pd.date_range(start=fecha_inicio, end=fecha_fin, freq='MS')
serie_reservas = reservas.melt(var_name='Anio', value_name='Reservas').sort_values('Anio')['Reservas']
df_arima_reservas = pd.DataFrame({'Reservas': serie_reservas}, index=fechas)
df_arima_reservas.index.name = 'Tiempo'
\end{lstlisting}

  \item Definir una función para encontrar los parámetros ARIMA óptimos y aplicar el modelo SARIMA.
\begin{lstlisting}
def encontrar_parametros_arima(df, column_name):
    modelo = auto_arima(df[column_name], max_p=100, max_d=5, max_q=100, 
                        seasonal=True, m=12, trace=True)
    return modelo.order, modelo.seasonal_order

parametros_orden, parametros_orden_estacional = encontrar_parametros_arima(
    df_arima_reservas, 'Reservas')

model_sarima_reservas = SARIMAX(df_arima_reservas['Reservas'], 
                                order=parametros_orden, 
                                seasonal_order=parametros_orden_estacional)
modelo_fit_sarima_reservas = model_sarima_reservas.fit()
\end{lstlisting}

  \item Realizar predicciones con el modelo SARIMA ajustado.
\begin{lstlisting}
predicciones_sarima_reservas = modelo_fit_sarima_reservas.get_forecast(steps=12)
predicciones_sarima_reservas = predicciones_sarima_reservas.predicted_mean
\end{lstlisting}

  \item Generar y guardar un gráfico de líneas con las reservas bancarias por anio.
\begin{lstlisting}
plt.figure(figsize=(15, 8))
sns.lineplot(data=df_arima_reservas.reset_index(), x='Tiempo', y='Reservas', 
             marker='o', palette=sns.color_palette("husl", len(df_arima_reservas.columns)))
plt.gca().xaxis.set_major_formatter(mdates.DateFormatter('%b %Y'))
plt.xticks(rotation=45)
plt.title('Reservas Bancarias por ANIo')
plt.xlabel('Mes')
plt.ylabel('Reservas en millones de quetzales')
plt.legend(title='ANIo', loc='upper left', bbox_to_anchor=(1, 1))
\end{lstlisting}
\end{itemize}



\subsection{Anexo III}
\subsection{Demostración de desigualdad.}\label{demostracion derivada}
\begin{align}\label{desigualdad anexo}
    \frac{d\mu}{dt} &= \frac{D}{I}\left( \frac{(1+t)^{2p} - (1+t)^p-pt(1+t)^{p-1}}{((1+t)^p-1)^2} \right) \geq 0.
\end{align}
Esta demostración se realizará utilizando Inducción Matemática. Como caso base, tomemos un plazo $p=6$. 
\newpage
\textbf{Caso base. } \\ 
En primera instancia, notemos que 
\begin{align*}
    \cdot \frac{D}{I((1+t)^p-1)^2} &\geq 0,
\end{align*}
por lo que basta mostrar que el numerador de \ref{desigualdad anexo} es no negativo. Para ello, notemos que mostrar que
\begin{align*}
    (1+t)^{2p} - (1+t)^p-pt(1+t)^{p-1} &\geq 0
\end{align*}
es equivalente\footnote{Divídase la ecuación entre $(1+t)^{p-1}$} a mostrar que
\begin{align}\label{ecuacion equivalente}
    (1+t)^{p+1} - (1+t) -pt &\geq 0.
\end{align}
Sustituyendo $p=6$ en \ref{ecuacion equivalente} llegamos a 
\begin{align*}
    (1+t)^5 &\geq 6t+1,
\end{align*}
lo cual es verdadero. Asumimos que para $p= k$ se cumple
\begin{align}\label{ecu}
    (1+t)^{k-1} \geq kt +t +1 = t(k+1) + 1.
\end{align}
Multiplicando \ref{ecu} por $(1+t)$ llegamos a 
\begin{align}\label{ecu2}
    (1+t)^k &\geq t(k+1)(1+t) + t + 1.
\end{align}
Por otro lado,  
\begin{align*}\label{ecu2}
    kt^2+t^2 &\geq 0 \\
    \Rightarrow kt^2+t^2+kt+2t+1 &\geq kt+2t+1 \\ 
    \Rightarrow (1+t)(k+1)t &\geq t(k+2)+1.
\end{align*}
Por \ref{ecu} y \ref{ecu2} llegamos a 
\begin{align*}
    (1+t)^k &\geq t(k+2)+1,
\end{align*}
lo que concluye la prueba. \\

\subsection{Demostración de convexidad.} \label{demostracion convexidad}
En esta sección de los anexos, demostraremos que
\begin{align*}
    S &= \left\lbrace x \in \mathbb{R} \hspace{0.15cm} | \hspace{0.15cm} \mu(x) \leq \gamma - RCI \right\rbrace
\end{align*}
es convexo, donde $\mu(x)$ está definida como en \ref{mu}.\\
Notemos que $S\neq \emptyset$ pues $t\in S$, entonces podemos tomar $x_1 \leq x_2 \in S$.  Luego, 
\begin{align*}
   x_1 \leq tx_2 + (1-t)x_1 \leq x_2, \text{ para } 0\leq t \leq 1.
\end{align*}
Usando el hecho que $\mu(x)$ es creciente y que $x_2 \in S$ llegamos 
\begin{align*}
    \mu(x_1) \leq \mu(tx_2 + (1-t)x_1) \leq \mu(x_2) \leq \gamma - RCI,
\end{align*}
lo que implica que 
\begin{align*}
    \mu(tx_2 + (1-t)x_1) \leq \gamma - RCI \Rightarrow tx_2 - (1-t)x_1 \in S,
\end{align*}
lo que concluye la prueba.


% poner anexos con codigo para generar datos aleatorios?