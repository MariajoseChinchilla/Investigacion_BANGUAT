Las reservas bancarias, la gestión de liquidez y la implementación de políticas monetarias, en conjunto con el riesgo de crédito, tienen incidencia en la valoración de activos, los precios para créditos y las estrategias de gestión de riesgo en las instituciones financieras. La interacción de estos elementos  puede influir en las condiciones crediticias, por lo que es necesaria una gestión prudente y estratégica por parte de los bancos para cuidar la estabilidad del sistema financiero, y así promover el crecimiento económico sostenible  y la mitigación de vulnerabilidades financieras. \\ 

Ante el paradigma de la constitución de reservas bancarias, en Guatemala fue aprobada una normativa, la Normativa JM47-2022, que busca dar directices para los créditos de forma tal que se le solicita a las entidades bancarias, reservas un porcentaje de los montos que presten, de acuerdo a la mora que el créditohabiente tenga dentro de la institución a la que solicita el crédito y en otras instituciones bancarias, así como el tipo de crédito que se esté otorgando. Esta normativa claramente pone sobre la mesa la problemática de dar precios justos, inteligentes y humanos a los créditos.  \\

Este trabajo de investigación busca, desde un enfoque matemático, financiero y social, proponer dos resultados importantes: por un lado, se propone un modelo matemático avanzado que permita a las instituciones bancarias, encontrar un intervalo de posibles precios para sus créditos, de manera tal que puedan asegurar la maximización de su rentabilidad, mientras cuidan de la salud financiera de sus créditohabientes y garantiza la correcta constitución de las reservas bancarias; y por otro lado, se demuestra un teorema que garantiza que, siempre que la institución bancaria defina sus necesidades financieras de forma lógica y utilice el modelo propuesto por este trabajo para encontrar el intervalo de posibles precios para un crédito, es posible construir una función de personalización de precio. \\

Con el modelo matemático y el teorema que se presentan, se busca sugerir al sistema bancario de Guatemala, una forma de controlar los precios para que las tasas de crédito sean saludables para la población, y al mismo tiempo, las instituciones bancarias puedan seguir operando de forma rentable. \\

Como estudio preliminar, se presenta un análisis descriptivo de las situación de las reservas bancarias en Guatemala desde 2018 y una proyección utilizando un modelo autorregresivo integrado de media móvil estacionaria (SARIMA), para evaluar qué condiciones futuras podrían esperarse. Esto último, con el fin de justificar y/o entender el porqué de una normativa como la JM47-2022. Finalmente, se presenta el modelo matemático y el teorema mencionado. Se le pide al lector, que sea paciente con la lectura, puesto que el valor más grande de este trabajo se presenta en las últimas secciones del mismo.